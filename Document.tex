\documentclass{scrartcl}
\usepackage{listings}
\usepackage{iftex}
% Pakete u.a. für die Darstellung von Umlauten/Sonderzeichen
\ifPDFTeX
   \usepackage[utf8]{inputenc}
   \usepackage[T1]{fontenc}
   \usepackage{lmodern}
\else
   \ifXeTeX
     \usepackage{xltxtra}
   \else 
     \usepackage{luatextra}
   \fi
   \defaultfontfeatures{Ligatures=TeX}
\fi

\lstdefinelanguage{gf}
{
  morekeywords={abstract, flags, cat, fun, incomplete, concrete, of, open, in, lincat, lin, resource, param, oper, variants, table, interface, instance, def, data, lindef, printname,},
  sensitive=false,
  morecomment=[l]{--},
  morestring=[b]",
  stringstyle={\textit}
}
\lstset{basicstyle=\ttfamily}


\title{Grammatical Framework for Python programmers}
\begin{document}
\maketitle
\section{Types}
\subsection{Dynamic typing in Python}
\subsection{Static typing in GF}
\begin{itemize}
\item Strings vs. Token lists
\item Function types
\end{itemize}
\section{Context-free Grammars}
\subsection{Context-free Grammars in NLTK}
\subsection{Context-free Grammars in GF}
\section{Step beyond Context-freeness: Tables and Records}
\subsection{In Python}
\subsection{In GF}
\section{Smart paradigms}
\subsection{Smart paradigms in Python}
\subsection{Smart paradigms in GF}
\section{Other problems}
\begin{itemize}
\item compile-time vs. run-time strings
\item ...
\end{itemize}

\end{document}
