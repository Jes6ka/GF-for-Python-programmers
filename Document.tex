\documentclass{scrartcl}
\usepackage{listings}
\usepackage{iftex}
% Pakete u.a. für die Darstellung von Umlauten/Sonderzeichen
\ifPDFTeX
   \usepackage[utf8]{inputenc}
   \usepackage[T1]{fontenc}
   \usepackage{lmodern}
\else
   \ifXeTeX
     \usepackage{xltxtra}
   \else 
     \usepackage{luatextra}
   \fi
   \defaultfontfeatures{Ligatures=TeX}
\fi

\lstdefinelanguage{gf}
{
  morekeywords={abstract, flags, cat, fun, incomplete, concrete, of, open, in, lincat, lin, resource, param, oper, variants, table, interface, instance, def, data, lindef, printname,},
  sensitive=false,
  morecomment=[l]{--},
  morestring=[b]",
  stringstyle={\textit}
}
\lstdefinelanguage{cf}
{
  morekeywords={::=,.,},
  sensitive=false,
  morecomment=[l]{--},
  morestring=[b]",
  stringstyle={\textit}
}
\lstset{basicstyle=\ttfamily}

\newtheorem{exercise}{Exercise}

\title{Grammatical Framework for Python programmers}
\begin{document}
\maketitle
\section{Types}
\subsection{Dynamic typing in Python}

Basic types in Python consist of Strings and Numbers (Integer, Float, Complex). A special case are boolean (logical expression) that will be explained later.
You can ask Python to give you the type for expressions. These types are automatically inferred, we don't even have to help Python to figure out what type
a variable has, as can be seen in the last example below.

\begin{verbatim}
>>> type(3)
<class 'int'>
>>> type(3.0)
<class 'float'>
>>> type("Foo")
<class 'str'>
>>> type(complex('1+2j')) # Is that still considered basic?
<class 'complex'>
>>> a=3
>>> type(a)
<class 'int'>
\end{verbatim}

These basic types can be used as the form compound types. Compound types are among others Lists, Tuples and Dictionaries.
Thse again can be part of other compound types as well, e.g. list containing lists as elements. Python does not really enforce
that all elements of a list have the same type(!) as you can see in the examples. To access elements in compound objects we
can use the \texttt{[]} operator.

\begin{verbatim}
>>> type([])
<class 'list'>
>>> type([1,2,3])
<class 'list'>
>>> type([1,2,"foo","bar"])
<class 'list'>
<class 'list'>
>>> type(())
<class 'tuple'>
>>> type((1,2))
<class 'tuple'>
>>> type((1,2,"foo"))
<class 'tuple'>
>>> type({})
<class 'dict'>
>>> type({'foo':1,'bar':2})
<class 'dict'>
>>> type({'foo':1,'bar':'baz'})
<class 'dict'>
\end{verbatim}

A special case in Python are truth values. Python does not just have a boolean datatype but interprets certain values of other types as truth values.
The values considered \textbf{false} are: 
\begin{itemize}
\item \texttt{None} i.e. the empty object
\item \texttt{False} i.e. the logical constant
\item \texttt{zero} of any numeric type, e.g. \texttt{0, 0.0, 0j}.
\item any empty sequence, e.g.\texttt{'', (), []}, i.e. empty string, tuple, or list.
\item any empty mapping, for example, \texttt{\{\}}.
\item instances of user-defined classes, if the class defines a \_\_bool\_\_() or \_\_len\_\_() method, when that method returns the integer zero or bool value False.
\end{itemize}
Everything else is considered \textbf{true}.

Another interesting group of datatypes are enumeration types where you define a type by listing all possible values. In Python enumerable types are objects of class \texttt{enum} as can be seen in the next example. They can also for example be used as keys in dictionary. That gives us a way to express a mapping from grammatical number and case to a word form for german nouns.

\begin{verbatim}
>> class Number(Enum):
...   Sg = 1
...   Pl = 2
... 
>>> man={Number.Sg:"man",Number.Pl:"men"}
>>> man[Number.Sg]
'man'
>>> class Case(Enum):
...   Nom = 1
...   Gen = 2
...   Dat = 3
...   Acc = 4
... 
>>> mann={Number.Sg:{Case.Nom:"Mann", 
...                  Case.Gen:"Mannes", 
...                  Case.Dat:"Mann", 
...                  Case.Acc:"Mann"},
...       Number.Pl:{Case.Nom:"Männer", 
...                  Case.Gen:"Männer", 
...                  Case.Dat:"Männern", 
...                  Case.Acc:"Männern"}
... }
>>> mann[Number.Sg][Case.Gen]
'Mannes'
\end{verbatim}

\begin{exercise}
  If you haven't done so before, try the type() function in Python on different values and compare the output. 
\end{exercise}
One problem with this aproach is that Python does not enforce that we define mappings for all possible values which can be prone to errors. In the next example we only define values for some of the keys and then try to access undefined values which leads to an error.

\begin{verbatim}
>>> mann={
...   Number.Sg:{
...     Case.Nom:"Mann"
...   },
...   Number.Pl:{
...     Case.Dat:"Männern"
...   }
... }
>>> mann[Number.Sg][Case.Gen]
Traceback (most recent call last):
  File "<stdin>", line 1, in <module>
KeyError: <Case.Gen: 2>
\end{verbatim}

In addition we will have a look at functions. Functions in Python are usually defined and given a name. But Python also supports so-called
anonymous functions or lambda expressions. Below the successor function is defined in two differnt ways
\begin{verbatim}
>>> def succ(x) :
...   return x+1
... 
>>> type(succ)
<class 'function'>
>>> succ2 = lambda x : x+1
>>> type(succ2)
<class 'function'>
>>> type(lambda x: x+1)
<class 'function'>
\end{verbatim}

\subsection{Static typing in GF}

Types in GF are a bit more complicated. Due to the separation in abstract modules, concrete modules and resource modules we have different types that are important in different situations. To keep it simple for the beginning we will just look at resource modules for the moment. Here we can use all the different types available in GF. These incluse Token lists (Str), Integers, Floats, and Bool. Most of them are defined in a module called Predef but Bool is defined in the module Prelude. Anonymous function types (lambdas) are standard types in GF and in resource modules the only way to define functions.

\lstinputlisting[language=gf]{src/SimpleTypes.gf}

As you can see in the examples above, for each variable we first give the type for each of the variable. That is best practice but for most of these simple cases it is not really necessary because GF can infer the type automatically. The only case where the type has to be given is the function.

Also there is a difference between the operators \texttt{+} and \texttt{++}. The reason for that is that GF has a distinction between strings and string tuples as well as between compiletime strings and runtime strings.

Finally the function is esentially a lambda expression like the one used in Python. The difference is that the keyword \texttt{lambda} is replace by a backslash (\texttt{\textbackslash}) and the colon that separates the parameter from the function body is replaced by an arrow (\texttt{->}).

\begin{verbatim}
> cc s
"foobar"
0 msec
> cc st
"foo" ++ "bar"
0 msec
> cc b
Prelude.False
0 msec
> cc i
42
0 msec
> cc f
23.5
0 msec
> cc bar "foo"
"foobar"
0 msec
\end{verbatim}

Compound types are tables, records and function types
\section{Context-free Grammars}
\subsection{Context-free Grammars in NLTK}
\begin{exercise}
  Write a context-free grammar accounting for the following sentences in Italian:
  ...
  Try to parse the following strings:

  The first one should be accepted and the second one rejected.
\end{exercise}
\subsection{Context-free Grammars in GF}
\lstinputlisting[language=cf]{src/GF-CF.cf}
\lstinputlisting[language=cf]{src/GF-CF2.cf}
\lstinputlisting[language=cf]{src/GF-CF3.cf}
\begin{exercise}
  Write the same grammar from the previous task in GF. Generate all trees accounted for by the grammar. Can you guess how many trees will be
  generated from a grammar?
\end{exercise}
\section{Step beyond Context-freeness: Tables and Records}
\subsection{In Python}
\subsection{In GF}
\section{Smart paradigms}
\subsection{Smart paradigms in Python}
\begin{exercise}
  Implement a function that takes a string of a noun and generates the regular noun paradigm for English as a dictionary of dictionaries. Also define all necessary grammatical features as enumeration types
\end{exercise}
\subsection{Smart paradigms in GF}
\section{Other problems}
\begin{itemize}
\item compile-time vs. run-time strings
\item ...
\end{itemize}

\end{document}
